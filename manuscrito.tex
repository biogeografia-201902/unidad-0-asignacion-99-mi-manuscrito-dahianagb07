\documentclass[11pt,]{article}
\usepackage[left=1in,top=1in,right=1in,bottom=1in]{geometry}
\newcommand*{\authorfont}{\fontfamily{phv}\selectfont}
\usepackage[]{mathpazo}


  \usepackage[T1]{fontenc}
  \usepackage[utf8]{inputenc}



\usepackage{abstract}
\renewcommand{\abstractname}{}    % clear the title
\renewcommand{\absnamepos}{empty} % originally center

\renewenvironment{abstract}
 {{%
    \setlength{\leftmargin}{0mm}
    \setlength{\rightmargin}{\leftmargin}%
  }%
  \relax}
 {\endlist}

\makeatletter
\def\@maketitle{%
  \newpage
%  \null
%  \vskip 2em%
%  \begin{center}%
  \let \footnote \thanks
    {\fontsize{18}{20}\selectfont\raggedright  \setlength{\parindent}{0pt} \@title \par}%
}
%\fi
\makeatother




\setcounter{secnumdepth}{3}



\title{Hormigas UASD  }



\author{\Large Dahiana Guzmán Báez\vspace{0.05in} \newline\normalsize\emph{Estudiante, Universidad Autónoma de Santo Domingo (UASD)}  }


\date{}

\usepackage{titlesec}

\titleformat*{\section}{\normalsize\bfseries}
\titleformat*{\subsection}{\normalsize\itshape}
\titleformat*{\subsubsection}{\normalsize\itshape}
\titleformat*{\paragraph}{\normalsize\itshape}
\titleformat*{\subparagraph}{\normalsize\itshape}

\titlespacing{\section}
{0pt}{36pt}{0pt}
\titlespacing{\subsection}
{0pt}{36pt}{0pt}
\titlespacing{\subsubsection}
{0pt}{36pt}{0pt}





\newtheorem{hypothesis}{Hypothesis}
\usepackage{setspace}

\makeatletter
\@ifpackageloaded{hyperref}{}{%
\ifxetex
  \PassOptionsToPackage{hyphens}{url}\usepackage[setpagesize=false, % page size defined by xetex
              unicode=false, % unicode breaks when used with xetex
              xetex]{hyperref}
\else
  \PassOptionsToPackage{hyphens}{url}\usepackage[unicode=true]{hyperref}
\fi
}

\@ifpackageloaded{color}{
    \PassOptionsToPackage{usenames,dvipsnames}{color}
}{%
    \usepackage[usenames,dvipsnames]{color}
}
\makeatother
\hypersetup{breaklinks=true,
            bookmarks=true,
            pdfauthor={Dahiana Guzmán Báez (Estudiante, Universidad Autónoma de Santo Domingo (UASD))},
             pdfkeywords = {palabra clave 1, palabra clave 2},  
            pdftitle={Hormigas UASD},
            colorlinks=true,
            citecolor=blue,
            urlcolor=blue,
            linkcolor=magenta,
            pdfborder={0 0 0}}
\urlstyle{same}  % don't use monospace font for urls

% set default figure placement to htbp
\makeatletter
\def\fps@figure{htbp}
\makeatother

\usepackage{pdflscape} \newcommand{\blandscape}{\begin{landscape}}
\newcommand{\elandscape}{\end{landscape}}


% add tightlist ----------
\providecommand{\tightlist}{%
\setlength{\itemsep}{0pt}\setlength{\parskip}{0pt}}

\begin{document}
	
% \pagenumbering{arabic}% resets `page` counter to 1 
%
% \maketitle

{% \usefont{T1}{pnc}{m}{n}
\setlength{\parindent}{0pt}
\thispagestyle{plain}
{\fontsize{18}{20}\selectfont\raggedright 
\maketitle  % title \par  

}

{
   \vskip 13.5pt\relax \normalsize\fontsize{11}{12} 
\textbf{\authorfont Dahiana Guzmán Báez} \hskip 15pt \emph{\small Estudiante, Universidad Autónoma de Santo Domingo (UASD)}   

}

}








\begin{abstract}

    \hbox{\vrule height .2pt width 39.14pc}

    \vskip 8.5pt % \small 

\noindent Mi resumen


\vskip 8.5pt \noindent \emph{Keywords}: palabra clave 1, palabra clave 2 \par

    \hbox{\vrule height .2pt width 39.14pc}



\end{abstract}


\vskip 6.5pt


\noindent  \section{Introducción}\label{introducciuxf3n}

Las hormigas juegan un rol muy importante en el desarrollo de los
ambientes urbanos, estas pueden afectar de manera directa o indirecta a
muchos de lo seres vivos, como plantas y animales. Estas afecciones
pueden ser picaduras o mordeduras estas introduccen ácido fórmico en el
cuerpo de algunos animales causandole alergia. También dañan
edificaciones, alimentos, jardines y algunas pueden ser vectores de
agentes infecciosos (Klotz, Hansen, Pospischil, \& Rust, 2008; Robinson,
2005; Robinson \& others, 1996).

En la hispaniola existen 43 géneros y 147 especies (\emph{Ants of
hispaniola}, n.d.).

Mis preguntas de investigación son las siguientes:

¿Cuál es la distribución espacial entre los nidos edificado y
pavimentado que superan los 5 metros de distancia?

¿Influye el transito de humanos en la diversidad de hormigas?

¿Existe diferencia significativa en la densidad de nidos entre distintos
sustratos?

¿Qué tanto recambio de especies existe entre nidos de sustratos
herbáceos o áreas contruidas?

\section{Metodología}\label{metodologuxeda}

\ldots

\section{Resultados}\label{resultados}

\section{Discusión}\label{discusiuxf3n}

\section{Agradecimientos}\label{agradecimientos}

\section{Información de soporte}\label{informaciuxf3n-de-soporte}

\ldots

\section{\texorpdfstring{\emph{Script}
reproducible}{Script reproducible}}\label{script-reproducible}

\ldots

\section*{Referencias}\label{referencias}
\addcontentsline{toc}{section}{Referencias}

\hypertarget{refs}{}
\hypertarget{ref-AntWiki}{}
\emph{Ants of hispaniola}. (n.d.).
\url{http://www.antwiki.org/wiki/Ants_of_Hispaniola}.

\hypertarget{ref-klotz2008urban}{}
Klotz, J. H., Hansen, L. D., Pospischil, R., \& Rust, M. (2008).
\emph{Urban ants of north america and europe: Identification, biology,
and management}. Cornell University Press.

\hypertarget{ref-robinson2005urban}{}
Robinson, W. H. (2005). \emph{Urban insects and arachnids: A handbook of
urban entomology}. Cambridge University Press.

\hypertarget{ref-robinson1996urban}{}
Robinson, W. H., \& others. (1996). \emph{Urban entomology: Insect and
mite pests in the human environment.} Chapman \& Hall.




\newpage
\singlespacing 
\end{document}
