\documentclass[11pt,]{article}
\usepackage[left=1in,top=1in,right=1in,bottom=1in]{geometry}
\newcommand*{\authorfont}{\fontfamily{phv}\selectfont}
\usepackage[]{mathpazo}


  \usepackage[T1]{fontenc}
  \usepackage[utf8]{inputenc}



\usepackage{abstract}
\renewcommand{\abstractname}{}    % clear the title
\renewcommand{\absnamepos}{empty} % originally center

\renewenvironment{abstract}
 {{%
    \setlength{\leftmargin}{0mm}
    \setlength{\rightmargin}{\leftmargin}%
  }%
  \relax}
 {\endlist}

\makeatletter
\def\@maketitle{%
  \newpage
%  \null
%  \vskip 2em%
%  \begin{center}%
  \let \footnote \thanks
    {\fontsize{18}{20}\selectfont\raggedright  \setlength{\parindent}{0pt} \@title \par}%
}
%\fi
\makeatother




\setcounter{secnumdepth}{3}


\usepackage{graphicx,grffile}
\makeatletter
\def\maxwidth{\ifdim\Gin@nat@width>\linewidth\linewidth\else\Gin@nat@width\fi}
\def\maxheight{\ifdim\Gin@nat@height>\textheight\textheight\else\Gin@nat@height\fi}
\makeatother
% Scale images if necessary, so that they will not overflow the page
% margins by default, and it is still possible to overwrite the defaults
% using explicit options in \includegraphics[width, height, ...]{}
\setkeys{Gin}{width=\maxwidth,height=\maxheight,keepaspectratio}

\title{Descripción de Nidos de Hormigas UASD  }



\author{\Large Dahiana Guzmán Báez\vspace{0.05in} \newline\normalsize\emph{Estudiante, Universidad Autónoma de Santo Domingo (UASD)}  }


\date{}

\usepackage{titlesec}

\titleformat*{\section}{\normalsize\bfseries}
\titleformat*{\subsection}{\normalsize\itshape}
\titleformat*{\subsubsection}{\normalsize\itshape}
\titleformat*{\paragraph}{\normalsize\itshape}
\titleformat*{\subparagraph}{\normalsize\itshape}

\titlespacing{\section}
{0pt}{36pt}{0pt}
\titlespacing{\subsection}
{0pt}{36pt}{0pt}
\titlespacing{\subsubsection}
{0pt}{36pt}{0pt}





\newtheorem{hypothesis}{Hypothesis}
\usepackage{setspace}

\makeatletter
\@ifpackageloaded{hyperref}{}{%
\ifxetex
  \PassOptionsToPackage{hyphens}{url}\usepackage[setpagesize=false, % page size defined by xetex
              unicode=false, % unicode breaks when used with xetex
              xetex]{hyperref}
\else
  \PassOptionsToPackage{hyphens}{url}\usepackage[unicode=true]{hyperref}
\fi
}

\@ifpackageloaded{color}{
    \PassOptionsToPackage{usenames,dvipsnames}{color}
}{%
    \usepackage[usenames,dvipsnames]{color}
}
\makeatother
\hypersetup{breaklinks=true,
            bookmarks=true,
            pdfauthor={Dahiana Guzmán Báez (Estudiante, Universidad Autónoma de Santo Domingo (UASD))},
             pdfkeywords = {Ecología, nidos},  
            pdftitle={Descripción de Nidos de Hormigas UASD},
            colorlinks=true,
            citecolor=blue,
            urlcolor=blue,
            linkcolor=magenta,
            pdfborder={0 0 0}}
\urlstyle{same}  % don't use monospace font for urls

% set default figure placement to htbp
\makeatletter
\def\fps@figure{htbp}
\makeatother

\usepackage{pdflscape} \newcommand{\blandscape}{\begin{landscape}}
\newcommand{\elandscape}{\end{landscape}}


% add tightlist ----------
\providecommand{\tightlist}{%
\setlength{\itemsep}{0pt}\setlength{\parskip}{0pt}}

\begin{document}
	
% \pagenumbering{arabic}% resets `page` counter to 1 
%
% \maketitle

{% \usefont{T1}{pnc}{m}{n}
\setlength{\parindent}{0pt}
\thispagestyle{plain}
{\fontsize{18}{20}\selectfont\raggedright 
\maketitle  % title \par  

}

{
   \vskip 13.5pt\relax \normalsize\fontsize{11}{12} 
\textbf{\authorfont Dahiana Guzmán Báez} \hskip 15pt \emph{\small Estudiante, Universidad Autónoma de Santo Domingo (UASD)}   

}

}








\begin{abstract}

    \hbox{\vrule height .2pt width 39.14pc}

    \vskip 8.5pt % \small 

\noindent Mi resumen


\vskip 8.5pt \noindent \emph{Keywords}: Ecología, nidos \par

    \hbox{\vrule height .2pt width 39.14pc}



\end{abstract}


\vskip 6.5pt


\noindent  \section{Introducción}\label{introducciuxf3n}

Las hormigas juegan un rol muy importante en el desarrollo de los
ambientes urbanos, estas pueden afectar de manera directa o indirecta a
muchos de lo seres vivos, como plantas y animales. Estas afecciones
pueden ser picaduras o mordeduras estas introduccen ácido fórmico en el
cuerpo de algunos animales causandole alergia. También dañan
edificaciones, alimentos, jardines y algunas pueden ser vectores de
agentes infecciosos (Klotz, Hansen, Pospischil, \& Rust, 2008; Robinson,
2005; Robinson \& others, 1996).

Las hormigas pertenecen al reino Animalia, filo Arthropoda, clase
Insecta, orden Hymenoptera y se distinguen de los demas animales por por
pertener a una única familia Formicidae. Se conocen alrededor de 12,000
a 20,000 especies de hormigas en el mundo, estas son clasificadas en
subfamilias (Chacón de Ulloa et al., 2008). En la hispaniola existen 43
géneros y 147 especies (\emph{Ants of hispaniola}, n.d.).

Las hormigas son seres vivos muy peculiares y apesar de esto se reunen
en grupos de especies a los cuales se le llaman gremios, cada gremio
comparten similitudes diferentes, como aspectos de su biología,
preferencia de hábitats y nichos; ejemplo de un gremio que ocupa de
forma exclusiva un nicho son las cultivadoras de hongos, todas las
hormigas de la tribu Attini (Chacón de Ulloa et al., 2008).

En esta investigación tomamos en cuenta la Ecología de nidos en la
Universidad Autonoma de Santo Dominingo. El nido es la parte fundamental
de la sociedad de hormigas. Cerca del 80-90\% de los miembros de una
colonia pertenecen en el nido (Petal, 1978). La arquitectura de los
nidos es muy variada, todo depende las especies que habiten en el nido.
Existen generos de hormigas que habitan ante todo en el suelo,
hojarasca, troncos o incluso en otros animales, por ejemplo Wasmannia
auropunctata y Paratrechina fulva (I. Armbrecht \& Ulloa-Chacón, 2003,
Zenner-Polania (1990)).

La ubicación de los nidos depende de los factores ambientales como
temperatura y humedad, también depende de la facilidad de reclutamiento
de alimentos para poder sobrevivir y reproducirse exitosamente Bernstein
\& Gobbel (1979).

Los nidos de hormigas y su modo de distribución en el espacio nos dan
información complementaría en el estudio de la comunidad en si. Por esa
razón para realizar esta investigación se tomaron en cuenta las
siguientes preguntas:

1- ¿Cuál es la distribución espacial entre los nidos edificado y
pavimentado que superan los 5 metros de distancia?

2- ¿Influye el transito de humanos en la diversidad de hormigas?

3- ¿Existe diferencia significativa en la densidad de nidos entre
distintos sustratos?

4- ¿Qué tanto recambio de especies existe entre nidos de sustratos
herbáceos o áreas contruidas?

\section{Metodología}\label{metodologuxeda}

\emph{Área de Estudio}

El trabajo se realizó en el campo de la Universidad Autonoma de Santo
Domingo (UASD) (18 27 40 N, 69 55 02 W). Tiene un área aproximada de
375, 000 m. Limita al norte con la Av. José Contreras, al sur con la Av.
Correa y Cidrón, al este con la Av. Santo Tomás de Aquino, y al oeste
con la calle General Modesto Diaz. Posee una temperatura promedio anual
de 25.7 C. Eligimos está área porque tiene un fácil acceso y por poseer
diferentes tipos de sustratos o coberturas como herbaceos, dosel,
construido, edificados, no edificado ni cubierto, entre otros. Además en
esta área existe una gran diversidad de hormigas.

\begin{figure}
\centering
\includegraphics{uasd.jpg}
\caption{}
\end{figure}

\emph{Métodos}

Se seleccionarion un total de 11 parcelas establecidas para el campo de
la UASD con diferentes tipos de coberturas, construido, mobiliario,
suelo, herbáceos, no edificado ni cubierto. Los muestreos fueron
realizados desde el día 12 al 26 de octubre del año 2019.

Los materiales de campo fueron: frascos, alcohol etílico al 80\%,
pinceles de cerdas claras, papel vegetal para las etiquetas, chinográfo
para escribir, dispositivo Android para llenar los formulario de ODK
Collect. ODK es un conjuntos de herramientas de código libre que crea
formularios para poder recoger los datos en un dispositivo móvil y
enviarlos a un servido.

Como se menciono anteriormente esta investigación se basó en la ecología
de nidos. Para realizar la coleccón de datos se hizo un censo detallado
de nido en cada parcela, tomando datos dentro de la cobertura que le
corresponde a la parcela elegida. Luego se toman las coordenas de cada
nido, información ambiental y relación de flora asociada al nido.

En cada nido se colecto de 5 a 8 individuos, para hacerlo utilizamos un
pincel humedicido con alcohol etílico al 80\%, luego cada individuo de
un mismo nido se deposito en un mismo frasco, es decir. Se utilizo un
frasco por nido el cual estaba debidamente etiquetado en papel vegetal
el nombre del colector, fecha y hora, el número de la parcela y la
muestra (p\#m\#). El trabajo de campo fue realizado por dos personas,
una relleno el formulario de ODK y la otra colecto las hormigas. Por
último pero no menos importante, los formularios fueron enviados a un
servidor para ser evaluados.

Culminado con la recolección de datos del campo, el siguiente paso fue
realizar la identificación de cada individuo encontrado en cada nido.
Para esto se utilizo una lupa de modelo AmScope 3.5X-180X Inspection
Zoom Stereo Microscope +144-LED Light, pinzas, porta objetos, alcohol al
80\%, guía de identificación de AntWiki y llenar los formularios para
identificación de ODK.

\section{Resultados}\label{resultados}

Colocar imagenes de algunos de los generos encontrados

(mapa(`riqueza', filtusuario = `dahianagb07'))

\section{Discusión}\label{discusiuxf3n}

\section{Agradecimientos}\label{agradecimientos}

\section{\texorpdfstring{\emph{Script}
reproducible}{Script reproducible}}\label{script-reproducible}

\section*{Referencias}\label{referencias}
\addcontentsline{toc}{section}{Referencias}

\hypertarget{refs}{}
\hypertarget{ref-AntWiki}{}
\emph{Ants of hispaniola}. (n.d.).
\url{http://www.antwiki.org/wiki/Ants_of_Hispaniola}.

\hypertarget{ref-armbrecht2003little}{}
Armbrecht, I., \& Ulloa-Chacón, P. (2003). The little fire ant wasmannia
auropunctata (roger)(Hymenoptera: Formicidae) as a diversity indicator
of ants in tropical dry forest fragments of colombia.
\emph{Environmental Entomology}, \emph{32}(3), 542--547.

\hypertarget{ref-bernstein1979partitioning}{}
Bernstein, R. A., \& Gobbel, M. (1979). Partitioning of space in
communities of ants. \emph{The Journal of Animal Ecology}, 931--942.

\hypertarget{ref-chacon2008aspectos}{}
Chacón de Ulloa, P., Armbrecht, I., Lozano-Zambrano, F., Jiménez, E.,
Fernández, F., \& Arias, T. (2008). Aspectos de la ecología de hormigas
cazadoras en bosques secos colombianos. \emph{Sistemática, Biogeografía
Y Conservación de Las Hormigas Cazadoras de Colombia}.

\hypertarget{ref-klotz2008urban}{}
Klotz, J. H., Hansen, L. D., Pospischil, R., \& Rust, M. (2008).
\emph{Urban ants of north america and europe: Identification, biology,
and management}. Cornell University Press.

\hypertarget{ref-petal1978role}{}
Petal, J. (1978). Role of ants in ecosystems. \emph{International
Biological Programme}.

\hypertarget{ref-robinson2005urban}{}
Robinson, W. H. (2005). \emph{Urban insects and arachnids: A handbook of
urban entomology}. Cambridge University Press.

\hypertarget{ref-robinson1996urban}{}
Robinson, W. H., \& others. (1996). \emph{Urban entomology: Insect and
mite pests in the human environment.} Chapman \& Hall.

\hypertarget{ref-zenner1990biological}{}
Zenner-Polania, I. (1990). Biological aspects of the ``hormiga loca'',
paratrechina (nylanderia) fulva (mayr). \emph{Colombia. in: Vander Meer
RK, Jaffe K, Cedeno A (Ed) Applied Myrmecology: A World Perspective.
Westview Press, Boulder}, 290--297.




\newpage
\singlespacing 
\end{document}
